\documentclass[10pt,a4paper]{article}
\usepackage[a4paper,margin=0.5in]{geometry}
\usepackage{enumitem}
\usepackage{titlesec}
\usepackage{array}
\usepackage{booktabs}
\usepackage{mathptmx} % Times New Roman-like font
\usepackage{amsmath}
\usepackage[hidelinks]{hyperref}


\pagenumbering{gobble}
% Adjust line spacing for compactness
\renewcommand{\baselinestretch}{0.95}  % Adjust line spacing (default is 1.0)
\setlength{\parskip}{0.5em}  % Adjust paragraph spacing
% Formatting for sections
\titleformat{\section}{\bfseries\large}{}{0em}{}
\titleformat{\subsection}{\bfseries\small}{}{0em}{}
\begin{document}
% Header
\begin{center}
    \textbf{\LARGE \underline{ABDELRHAMAN KHALAFALLA}} \\ 
    \vspace{1mm}
    Akhalafalla5@gmail.com  | \href{https://www.linkedin.com/in/abdelrahman-khalafalla-79760a25b/}{https://www.linkedin.com/in/abdelrahman-khalafalla-79760a25b/}
\end{center}

% Education
\vspace{-10mm}
\section*{\vspace{3mm}\hrule\vspace{-10mm}}
\section{\underline{EDUCATION}}
\vspace{-5mm}
\textbf{Cyprus International University} \hfill \textbf{Haspolat, Lefkoşa, North Cyprus} \\
\textit{Bachelor of Science in Electrical and Electronic Engineering} \hfill \textbf{CGPA 3.07} \\
\textit{Electronics and Communication Track} \\
\textbf{\underline{Awards \& Honors:}} Faculty Of Engineering Honors List, CIU-WINDCOM First Place Developer\\
SET EVENT: CIU MATLAB CODEATHON BEST INNOVATIVE PROJECT AWARD\\
\\
\vspace{-12mm}
% Skills
\section*{\vspace{1mm}\hrule\vspace{-10mm}}
\section{\underline{WORKING KNOWLEDGE AND SKILLS}}
\vspace{-3mm}
\textbf{• Python} \hspace{60mm} \textbf{• LTspice} \hspace{36.5mm}\textbf{• Kreas}\\
\textbf{• HTML} \hspace{60mm} \textbf{• KiCad} \hspace{38.5mm}\textbf{• Pandas}\\
\textbf{• CSS} \hspace{64.2mm} \textbf{• Altium Designer} \hspace{24mm}\textbf{• NumPy}\\
\textbf{• JavsScript} \hspace{55mm} \textbf{•AutoCAD} \hspace{34.7mm}\textbf{• TensorFlow}\\

\vspace{-9mm}

% Projects
\section*{\vspace{2mm}\hrule\vspace{-10mm}}
\section{\underline{PROJECTS}}
\vspace{-3mm}
\textbf{AI Prosthetic Hand Controlled Via The Peripheral Nervous System} \hfill \textbf{February 2024 -- July 2024} \\
\begin{itemize}[leftmargin=*]
    \vspace{-8mm}
    \item Designed the 3D model of the Prosthetic Hand using TinkerCad software and Cubicon Single-Plus Printer to print all the required parts Using PLA printing Material for printing, ensuring it is wearable for amputation people.
    \vspace{-2mm}
    \item Developed an Signal aqcuistion system using the MyoWare 2.0 Muscle Sensor to detect the EMG signals from the targeted muscles for different gestures and convert them to electrical signals to control the prosthetic hand via Servo motors.
    \vspace{-2mm}
    \item Aquiring EMG signals for 9 different gestures from different voulenteers to train the Machine Learning model, and testing the model on different voulenteers to ensure the model's accuracy. 
     \vspace{-1mm}
     \item Developed an AI model using TensorFlow to classify the EMG signals and control the prosthetic hand to perform the desired gesture.
     
\end{itemize}
\vspace{-1mm}
\textbf{Techno-Economic Analysis of Standalone off-Grid Smart Parking lot in a Smart City} \hfill \textbf{March 2023 -- May 2023} \\
\begin{itemize}[leftmargin=*]
    \vspace{-8mm}
    \item Developed a sustainable solution to address the global energy demand, Foucsing on a smart parking area with EV
    chargers, contributing to a 100\%  sustainable smart city initiative.
    \vspace{-2mm}
    \item Designed a standalone off-grid solar system to power the parking lot, using HOMER Grid Software to optimize the system's size and components to maximize efficiency and minimize costs.
    \vspace{-2mm}
    \item Conducted a techno-economic analysis to evaluate the system's performance, considering factors such as the initial investment, operational costs, and payback period to determine the system's feasibility and sustainability.
\end{itemize}
\vspace{-2mm}
\textbf{Frequency Response Of 2-Stage Common Emitter Amplifier} \hfill \textbf{May 2023 -- June 2023} \\
\begin{itemize}[leftmargin=*]
    \vspace{-8mm}
    \item Designed a 2-STAGE Common Emitter Amplifier Circuit using LTSPICE Simulator, Foucsing on achieving stable performance for both DC and AC operations to understand how coupling, bypass, wiring, And parasitic capacitors effects  the amplifier's performance.
    \vspace{-2mm}
    \item Conducted AC analysis by applying different AC signals to evaluate the amplifier's behavior across different frequencies, measured its gain in decibels (dB) to determine the amplification factor, and generated Bode plots to visualize magnitude and phase responses over the frequency range.
   \vspace{-2mm}
    \item Applied Miller's theorem to analyze the effect of feedback capacitances, identified low-pass and high-pass filter characteristics, and determined the locations of poles and zeros to understand the frequency-dependent behavior of the circuit.
    \vspace{-2mm}
\end{itemize}
\vspace{-2mm}
\textbf{Line Following Robot} \hfill \textbf{December 2023 -- January 2024} \\
\begin{itemize}[leftmargin=*]
    \vspace{-8mm}
    \item  Designed a Line Following Robot that autonomously tracks lines using a PID controller for precise navigation, integrating the QTR-8RC Sensor Array to detect lines and determine the robot's position relative to the path.
    \vspace{-2mm}
    \item Conducted tests on different track configurations and environmental conditions to ensure robust performance.
   \vspace{-2mm}    
    \item Combined robotics mechanics with advanced control theory to enable effective autonomous navigation.
    \vspace{-9mm}
\end{itemize}
% Leadership Experience & Activities
\section*{\ \vspace{0mm}\hrule\vspace{-10mm}}
\section{\underline{WORK EXPERICENCE}}
\vspace{-3mm}
\textbf{Turkish Cyprus Electrical Corporation} \hfill \textbf{August 2023 -- October 2023} \\
\vspace{-8mm}
\begin{itemize}[leftmargin=*]
    \item Worked in the Power System Department, where I learned about the power system structure and quality measurment techniques for power systems, applying the symmetrical and unsymmetrical faults analysis to the power system to see the faults that may occur in the power system.
    \vspace{-2mm}
    \item Automated an entire production line using PLC Programming language, in order to increase the production efficiency and reduce the human error.
    \vspace{-2mm}
    \item Maintained and repaired the electrical equipment in the production line to ensure the production line's efficiency.
     \vspace{-2mm}
\end{itemize}
\vspace{340mm}

\textbf{TurkCell} \hfill \textbf{May 2023 -- July 2023} \\

\vspace{-8mm}
\begin{itemize}[leftmargin=*]
    \item Worked in Switch Department, where I learned about the GSM network structure and quality measurment techniques for GSM such as Frequency Division Multiple Access (FDMA) and Time Division Multiple Access (TDMA).
    \item learned about the wirless communication and the different types of antennas used in the GSM network.
    \item Maintained and repaired the electrical equipment in the Switch Department to ensure the Switch Department's efficiency.
    \item Learned about the different types of antennas used in the GSM network and how to maintain and repair them.
    \item Assisted in the installation and configuration of network equipment, ensuring optimal performance and reliability.
    \vspace{-2mm}
\end{itemize}
\vspace{-2mm}



\end{document}
